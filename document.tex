\documentclass[10pt,twocolumn]{article}

% use the oxycomps style file
\usepackage{oxycomps}

% usage: \fixme[comments describing issue]{text to be fixed}
% define \fixme as not doing anything special
\newcommand{\fixme}[2][]{#2}
% overwrite it so it shows up as red
\renewcommand{\fixme}[2][]{\textcolor{red}{#2}}
% overwrite it again so related text shows as footnotes
%\renewcommand{\fixme}[2][]{\textcolor{red}{#2\footnote{#1}}}

% read references.bib for the bibtex data
\bibliography{references}

% include metadata in the generated pdf file
\pdfinfo{
    /Title (Ethical Implications in Solar-Powered Vehicle Energy Modeling)
    /Author (Julia Chun)
}

% set the title and author information
\title{Ethical Implications in Solar-Powered Vehicle Energy Modeling}
\author{Julia Chun}
\affiliation{Occidental College}
\email{jchun2@oxy.edu}

\begin{document}

\maketitle

\section{Introduction}
Predicting the energy needs for a solar powered electric vehicle assures promotion for sustainability in transportation. However, this project raises ethical concerns. The ethical concerns surrounding solar-powered electric vehicle energy modeling are substantial and varied and presents significant challenges to the continuation of this project. By critically examining issues related to data bias, accessibility, power dynamics, transparency, privacy, security, upkeep and maintenance, and technological solutionism, this essay demonstrates that the ethical concerns presents challenges which renders the project unsustainable with the absence of ethical frameworks and mitigation strategies. 
\section{Solar Energy}
This project involves predicting the energy needs for an electric vehicle which utilizes solar energy as its main energy source. Solar energy was determined to be used as the energy source for this model as it is an emerging renewable energy source which aims to reduce carbon emissions and fight against climate change. According to MIC, in March, The Houston Chronicle reported that solar energy was the fastest growing energy sector in the United States. This is partially due to the affordability and accessibility solar power has become over the years. Furthermore, the cost of solar panels has dropped nearly 70% since 2014. Although this energy source is promising in providing sustainable energy, there are various ethical concerns associated with this emerging energy source.    
\subsection{Solar Panels}
At the forefront of these concerns is the environmental and ethical impact of solar energy infrastructure. Solar panels are the necessary instruments that harness the power of the sun to convert into mechanical energy. However, the production of solar panels raises concerns associated with the manufacturing and disposing of the solar panels. Solar panels requires silicon, which is derived from sand  and rare metals. Hence, the manufacturing process encompasses mining, transport and refining which will inevitably contribute to a carbon and ecological footprint. According to BigSolar cooperation, "Coal fuels 62% of the electricity used for solar PV manufacturing ", which is a contributing source to the carbon footprint.   There are also ethical concerns related to supply chain issues and forced labor. According to the organization FRDM, "95% of solar modules are made from the same primary material, solar-grade polysilicon. 45% of the world’s solar-grade polysilicon supply comes from the Uyghur Autonomous Region, Xinjiang". Additionally, the source mentions that the mines that produce some of the other key materials needed for solar infrastructure are in countries without stringent environmental and labor standards. Much of China’s polysilicon production occurs in its northwestern provinces, including Xinjiang. This region faces scrutiny for human rights abuses and systemic detention of the Uyghur people. Furthermore, despite the Chinese government denial of these charges, an increasing number of reports claim that Uyghurs are forced to produce wind and solar components. Disruptions in this supply chain would severely threaten and restrict solar energy development worldwide, presenting challenges for utility companies stiving to meet their customers’ needs through solar power. 
Additiionally, another concern arises from the complexity of the solar panel supply chain where the origin of solar-grade polysilicon can be buried. Consumers may mistakenly assume that they are purchasing “above board” solar panels made by a local manufacturer only to dsicover that their solar modules might originate another country sourcing polysilicon from Xinjiang. The numerous steps in this supply chain can make identifying and addressing the risks even more difficult to address potential issues. 
\subsection{Transition of using Solar Energy}
As the world is undergoing significant transformation characterized by a decline in fossil fuel usage with a surge in renewable energy production and consumption,there is a reorganization of redistributing power and wealth. The transition towards this emerging renewable energy source could be scrutinized for its ethical concerns. The shift towards renewable energy sources and the move away from fossil fuels risks leaving behind workers in the fossil fuel industry leading to economic disenfranchisement and social unrest .Furthermore, the rapid expansion of solar energy technologies may exacerbate existing inequalities, as access to renewable energy solutions could remain unequal across various regions and socioeconomic groups. Politically, this transition could contribute to power struggles and geopolitical tension. The reliance on rare earth metals for the solar panel production could intensify competition among nations for access to these resources, leading to conflicts over regions where these resources are more abundant. Furthermore ,the geopolitical implications of energy independence and security could be pronounces as countries aim to reduce reliance on fossil fuel imports. These geopolitical dynamics could shape international relations and influence political struggles. 
\subsection{Electric Vehicle}
In addition to the ethical concerns surrounding the adoption of solar energy in this project, the integration of electric vehicles in this project introduces additional ethical concerns. Similar to the embrace of solar energy, the integration of electric vehicles in the transportation sector signifies a shift towards a more sustainable mode of transportation. However, this transition presents new ethical dilemmas.   
\subsection{Environmental Impact of mining and processing of minerals and metals required to produce electric vehicle systems}
Similar to the environmental impacts and ethical concerns associated with solar energy infrastructure, the extraction of core components for electric vehicles also raises concern for significant consequences. In many regions, mining for materials such as cobalt or lithium necessary for electric vehicle batteries could result in environmental degradation and poses risks to local communities. Unlike traditional oil and gas extraction, which often relies on skilled laborers, mining for these materials could involve unskilled laborers who face poor working conditions and are vulnerable to exploitation similar to the ethical concerns of working conditions of mining resources of solar panels. Additionally, mining activities can strain water resources and lead to toxic waste, causing pollution and ecosystem destruction. Furthermore, although electric vehicles are more sustainable during their operational phase, their economic viability depends on factors such as charging infrastructure, energy sources, and long-distance travel. The debate of where and how much value is added in the supply chain for electric vehicle components also arises. This raises concerns about the distribution of benefits and risks across different countries and communities. However, implementing measures to address these concerns, such as investing in cleaner extraction practices and promoting fair labor conditions, could increase the cost of electric vehicles and potentially delay the transition away from carbon-consuming vehicles. Despite efforts to reduce emissions associated with EV battery manufacturing, challenges remain in ensuring sustainability across the entire supply chain. As The Guardian's Nina Lakhani reported "Lithium extraction has a track record of land and water pollution, ecosystem destruction and violations against Indigenous and rural communities." There is also the  debate present over extraction versus development in the supply chain for electric vehicle components. This raises important considerations about where value is added and who bears the environmental and health risks. The allocation of greater investment in processing and fabrication steps in countries where raw materials are mined or extracted could distribute more value to those assuming environmental risks. Additionally, such measures could undercut promises of domestic rejuvenation, which could then hinder efforts to drive economic growth through the energy transition. The global supply chains for electric vehicle components also present challenges, as highlighted by Siddarth Kara's observation that China dominates key aspects of the supply chain, including cobalt mining and battery manufacturing. Furthermore, geopolitical tensions, such as efforts to reduce reliance on Russian oil income, could complicate efforts to decouple economies. 
\subsection{toxic waste of EV batteries}

The consequences of toxic waste from electric vehicle batteries are significant and multifaceted. Improper disposal of electric vehicle batteries could lead to environmental pollution, posing risks to ecosystems, wildlife, and human health. Toxic chemicals and heavy metals present in batteries, such as lithium, cobalt, and nickel, can be exposed into soil and water sources, contaminating them and potentially causing long-term harm to local communities and ecosystems. Furthermore, the accumulation of toxic waste from electric vehicle batteries exacerbates existing challenges in waste management and disposal infrastructure. Landfills may become overwhelmed with hazardous materials, increasing the risk of groundwater contamination and soil degradation. Incineration of batteries can release harmful emissions into the atmosphere, contributing to air pollution and respiratory illnesses in nearby communities. Furthermore, the global demand for electric vehicles is expected to continue rising, leading to a corresponding increase in battery production and waste generation. Without adequate recycling and disposal infrastructure in place, the accumulation of toxic battery waste will become a growing environmental and public health concern. Addressing the consequences of toxic waste from electric vehicle  batteries requires comprehensive policies and initiatives aimed at promoting sustainable battery recycling and disposal practices. 

\subsection{Accelerating Industry of Electric Vehicles}
The accelerating industry of electric vehicles presents various of ethical concerns which require careful consideration. One significant implication is the potential exacerbation of social and economic inequalities similar to the concerns present in the transition to solar energy. As EV adoption increases, there is a risk that only affluent individuals will have access to clean transportation options, leaving marginalized communities reliant on traditional fossil fuel-powered vehicles. This disparity in access to sustainable mobility could deepen existing inequities, particularly in areas where public transportation infrastructure is lacking. Moreover, the widespread adoption of EVs may lead to job displacement in industries related to traditional internal combustion engine vehicles, affecting workers' livelihoods and economic stability. Additionally, the shift towards EVs may place strain on existing energy grids and infrastructure, especially if charging infrastructure development lags behind the increasing number of electric vehicles on the road. 

\section{Path Planning ethical concerns}
Path planning optimization will be used in this project to predict optimal routes for the vehicle to follow  with the objective of collecting, conserving, and optimizing solar energy. However, according to the source Lexinter Law directory, there are two main ethical concerning regarding route planning technology: accessibility and environmental impact. The article mentions how route planning technology would make travel easier and optimal for many, and this may inadvertently ostracize these who cannot afford or do not have access to such tools. Lexinter Law also explains legal concerns of route planning technology. There is the question of legal liability when navigation errors lead to accidents. The article presents an example where if a mapping application provides incorrect directions and causes a user to have an accident, the company behind the route planner could potentially be held legally accountable.

\section{Ethical Concerns of Data and Data Collection}

\subsection{Data Bias}
This project involves collecting and analyzing data related to various factors such as solar radiance,vehicle energy consumption, battery capacity, traffic congestion, terrain, orientation of the car,  and more. Data bias is a pervasive concern in solar-powered vehicle energy modeling, stemming from the inherent biases present in datasets used for modeling. Biased data may lead to inaccurate predictions and reinforce existing inequalities. For instance, for this project, if the solar data collected came from regions where abundant sunlight or infrastructure is already well developed, this may not accurately represent the solar potential in region with less access to sunlight or fewer resources from solar energy development. This could perpetuate disparity in access to clean energy technologies. Furthermore, biased data inputs could result in models which disproportionately favor certain interests. For instance, if a model is generated representing an overestimation of energy generation potential of solar panel or underestimates the energy consumption of vehicles, unrealistic expectations of flawed decision-making with regards to vehicle design or infrastructure planning could be made. 
 
\subsection{Privacy and Consent}
Protecting individuals' privacy and ensuring data security are paramount considerations in solar-powered vehicle energy modeling projects. However, collecting, storing, and analyzing sensitive personal data raise privacy concerns and may expose individuals to risks of harm. Although data is being collected from external sources rather than self-collecting the data, there are still privacy and consent concerns and risks. Collecting and utilizing this data without adequate privacy safeguards could lead to concerns regarding profiling or surveillance. If data was collected without adequate consent or if individual's privacy rights were not respected during data collection continuation of using the data could lead to ethical violations and reputational damage for the project. Furthermore, there could also be challenges present regarding transparency. If the data collection methods of the data being used were not transparently documented, the validity and representatives of the data could be questions and hinder efforts to assess potential shortcomings or biases in the data. 

\section{Accessibility}
Ensuring equitable access to solar-powered vehicle energy modeling tools and resources is essential for promoting accessibilit. However, accessibility barriers, such as limited availability of data, software, and technical expertise, pose significant challenges. Marginalized communities and individuals with disabilities may face disproportionate barriers to access, perpetuating existing disparities in transportation and energy access. Despite efforts to improve accessibility, structural barriers and resource constraints hinder progress towards equitable access
\section{Power Dynamics}
Solar-powered vehicle energy modeling projects have the potential to either distribute power to communities or concentrate power in the hands of a few individuals or organizations. Decision-making processes, data ownership, and resource allocation can influence power dynamics within project ecosystems. However, addressing power imbalances requires systemic changes to democratize decision-making, promote community engagement, and redistribute resources equitably. Without fundamental shifts in power dynamics, solar-powered vehicle energy modeling projects may exacerbate existing inequalities and reinforce dominant power structures.
\section{Transparency}
Ensuring transparency and explainability in solar-powered vehicle energy modeling is essential for fostering trust, accountability, and public engagement. However, complex modeling algorithms, proprietary software, and opaque decision-making processes may hinder transparency efforts. Lack of transparency undermines public trust and confidence in modeling outcomes, hindering meaningful dialogue and collaboration. Despite calls for greater transparency, institutional and cultural barriers impede progress towards more open and accountable modeling practices.

\section{Maintenance}
Regarding if one wanted to expand upon this project, ensuring the long-term sustainability and effectiveness of this project would require ongoing maintenance efforts. However, challenges regarding limited resources, technological obsolescence, and shifting societal priorities may challenge this project's sustainability of continuation. Without adequate support and investment in maintenance activities, this project may fail to deliver intended benefits, undermining the ethical imperative of promoting sustainable transportation solutions.






\printbibliography

\end{document}
